%\documentclass[handout]{beamer}

% \documentclass{beamer}% for teacher version
\documentclass[handout]{beamer}% for student version

%\usepackage{pgfpages}
%\usepackage{handoutWithNotes}
%\pgfpagesuselayout{4 on 1 with notes}[a4paper,border shrink=5mm]

\usepackage{etex}
 
 
%\usepackage[ngerman, american, british]{babel} 
\usepackage[T1]{fontenc}
\usepackage[utf8]{inputenc} %%% latin input
%\DeclareUnicodeCharacter{00A0}{ }

%\usepackage{default}
\usetheme{Warsaw}
%\usepackage{authblk}
\usepackage{multicol}

\usepackage{subcaption}
%\usepackage[style=numeric,backend=biber]{biblatex}
% %\usepackage[style=numeric]{biblatex}
%\addbibresource{../bib.bib}
%\AtEveryCitekey{%
%\clearfield{title}%
%\clearfield{pagetotal}%
%}
% Make bracketized
%\renewcommand*{\thefootnote}{[\arabic{footnote}]}
%\makeatletter
% Remove superscript for footnotemark
%\def\@makefnmark{\hbox{{\normalfont\@thefnmark}}}
% Allow space to precede the footnote
%\usepackage{etoolbox}
%\patchcmd{\blx@mkbibfootnote}{\unspace}{}{}
%\makeatother
\setbeamertemplate{navigation symbols}{}%remove navigation symbols


\usepackage{textpos}
\usepackage[percent]{overpic}

\newcommand{\PreLim}
{\begin{tikzpicture}
% \node [opacity=0.1,rotate around={45:(0,0)}] (0,0) {\scalebox{2.0}{\textcolor{green}{Preliminary}}};
\end{tikzpicture}}
\newcommand{\PreLimP}{\put (50,0) {\PreLim}}
  
%\usepackage[printwatermark]{xwatermark}
%\newwatermark*[allpages,color=red!50,angle=45,scale=2,xpos=0,ypos=0]{DRAFT}

%\usepackage{struktex} %%% create structograms
\usepackage{booktabs} %%% scientific publication style tables
\usepackage{empheq}   %%% for boxed equations and alike
\usepackage{xcolor}   %%% for colored text and formulae
\usepackage{amsmath}  %%% math symbols needed
\usepackage{amsfonts} %%% math fonts needed
\usepackage{amssymb}  %%% maths symbols needed 
\usepackage{amsthm}   %%% theorem environments
%\usepackage{dsfont}   %%% for blackboard numbers and characters 
%\usepackage{mathrsfs} %%% fonts when speaking e.g. about Lagrangian density
%\usepackage{mathdots/mathdots}
\usepackage{pdfpages}
\usepackage{multirow} %%% merge rows in a table

% Default graphics paths
\graphicspath{{figs/}}

%use of ps ticks
% \usepackage[usenames,dvipsnames]{pstricks}
% \usepackage{epsfig}
% \usepackage{pst-grad} % For gradients
% \usepackage{pst-plot} % For axes
\usepackage{pgf,tikz}
\usetikzlibrary{shapes,arrows,positioning,calc}
\usetikzlibrary{tikzmark} % remember position for later

\ifx\du\undefined
  \newlength{\du}
\fi
\setlength{\du}{15\unitlength}

%\usepackage{fancybox} %%% more types of boxes 

%\usepackage[pdftex]{graphicx}       %%% CUSTOMIZATION: include all pictures
%\usepackage[draft]{graphicx} %%% choose draft option for faster compiling without pictures
%\graphicspath{{Figures/}}      %%% all pictures will reside there and in subdirs


%%%%%%%%%%%%%%%%%%%%%%%%%%%%%%%%%%%%%%%%%%%%%%%%%%
%%% acronym and notation section
%%% comment/delete this section, if you do not want to use these
%\usepackage[toc,style=list,acronym=true]{glossary} %%% this package has to go AFTER the hyperref package!
%\setacronymnamefmt{\gloshort}
%\setacronymdescfmt{\glolong}
%\input{Misc/acronyms-definitions.tex}
%\makeacronym
%
%\newglossarytype{notation}{not}{ntn}[style=list]
%\newcommand{\notationname}{Notation}
%\setglossary{type=notation,glsnumformat=ignore,glodelim={.}}
%\makenotation
%%% end of acronuyms and notation
%%%%%%%%%%%%%%%%%%%%%%%%%%%%%%%%%%%%%%%%%%%%%%%%%%

\newcommand\FourQuad[4]{%
    \begin{minipage}[b][.35\textheight][t]{.47\textwidth}#1\end{minipage}\vrule  \hfill%
    \begin{minipage}[b][.35\textheight][t]{.47\textwidth}#2\end{minipage} \\[0.25em]
    \hrule \hspace{0.001cm} \\[0.25em]
    \begin{minipage}[b][.35\textheight][t]{.47\textwidth}#3\end{minipage}\vrule \hfill
    \begin{minipage}[b][.35\textheight][t]{.47\textwidth}#4\end{minipage}%
    }

%%%%%%%%%%%%%%%%%%%%%%%%%%%%%%%%%%%%%%%%%%%%%%%%%%
%%% Selection of the fonts!! 
\usepackage[T1]{fontenc}
%\usepackage{helvet}
%%%%%%%%%%%%%%%%%%%%%%%%%%%%%%%%%%%%%%%%%%%%%%%%%%

\newenvironment<>{varblock}[2][.9\textwidth]{%
  \setlength{\textwidth}{#1}
  \begin{actionenv}#3%
    \def\insertblocktitle{#2}%
    \par%
    \usebeamertemplate{block begin}}
  {\par%
    \usebeamertemplate{block end}%
  \end{actionenv}}


%\usepackage{tikz}
%\usetikzlibrary{arrows,shapes}
% For every picture that defines or uses external nodes, you'll have to
% apply the 'remember picture' style. To avoid some typing, we'll apply
% the style to all pictures.
\tikzstyle{every picture}+=[remember picture]

% By default all math in TikZ nodes are set in inline mode. Change this to
% displaystyle so that we don't get small fractions.
\everymath{\displaystyle}

\setbeamercovered{transparent}



\title[Talys examples -- 1 (Friday)] % (optional, only for long titles)
{Introduction into the usage of TALYS}
\subtitle{Oslo-Berkeley-Stellenbosch-iThemba Summer School 2019}

\author[F. Zeiser]{Fabio Zeiser, UiO}

\date{29th November 2019} % (optional)
%{ND2016, Brugge}
%\subject{Computer Science}r}



%%%%%%%%%%%%%%%
%\usepackage[left=1.00in, right=1.00in, top=1.00in, bottom=1.00in]{geometry}
\usepackage{xpatch}
\usepackage{tcolorbox}
\tcbuselibrary{minted,skins,listings}

\tcbset{
        enhanced,
        boxrule=0.0pt,
        %fonttitle=\bfseries
       }

\newtcblisting{bashcodebg}[1][]{
    listing engine=minted,
    colback=bashcodebg,
    colframe=black!70,
    listing only,
    minted style=colorful,
    minted language=bash,
    minted options={linenos=true,numbersep=3mm,texcl=true,breaklines,#1},
    left=5mm,enhanced,
    overlay={\begin{tcbclipinterior}\fill[black!25] (frame.south west)
            rectangle ([xshift=5mm]frame.north west);\end{tcbclipinterior}}
}

\newtcblisting{bashcodebg2}[1][]{
    listing engine=minted,
    colback=bashcodebg,
    colframe=black!70,
    listing only,
    minted style=colorful,
    minted language=bash,
    minted options={linenos=false,numbersep=3mm,texcl=false,#1,xrightmargin=0.5\textwidth},
    left=0mm,enhanced,
    %overlay={\begin{tcbclipinterior}\fill[black!25] (frame.south west)
    %        rectangle (frame.north west);\end{tcbclipinterior}}
}

\definecolor{bashcodebg}{rgb}{0.85,0.85,0.85}

%%%%%%

\newlength{\mintedfboxsep}
\setlength{\mintedfboxsep}{0.2pt}

\newmintinline{python}{python3, framesep=0.1pt,bgcolor=bashcodebg}

\makeatletter
\newlength{\fboxrsep}
\setlength{\fboxrsep}{\fboxsep}

\newlength{\fboxlsep}
\setlength{\fboxlsep}{\fboxsep}

\newlength{\fboxtsep}
\setlength{\fboxtsep}{\fboxsep}

\newlength{\fboxbsep}
\setlength{\fboxbsep}{\mintedfboxsep}

\xpatchcmd{\minted@inputpyg@inline}{%
  \colorbox%
}{%
  \long\def\color@b@x##1##2##3%
  {\leavevmode
    \setbox\z@\hbox{\kern\fboxlsep{\set@color##3}\kern\fboxrsep}%
    \dimen@\ht\z@\advance\dimen@\fboxtsep\ht\z@\dimen@
    \dimen@\dp\z@\advance\dimen@\fboxbsep\dp\z@\dimen@
    {##1{##2\color@block{\wd\z@}{\ht\z@}{\dp\z@}\box\z@}}}%
  \colorbox%
}{\typeout{Success}}{\typeout{Failure}}
\makeatother

%%%%%%%





%%%%%%%%%

\begin{document}

\maketitle

%\begingroup
%\setbeamercolor{background canvas}{bg=gray!30}

\begin{frame}[fragile]{Basic example -- 1}

% \begin{frame}[fragile]
% \frametitle{Title}
% \framesubtitle{Subtitle}

Most simple input:\\
Requires only 4 keywords

 \begin{minipage}{0.5\textwidth}
\begin{bashcodebg}
\## Simple example code
projectile n
element    Fe
mass       56
energy     14.
\end{bashcodebg}
\end{minipage}
\newline
\newline
Run the input:\\
\pythoninline{talys <input> output}


\end{frame}

%%%%%%%%%%%%%%%%%%%%%%%%%%

\begin{frame}[fragile]{Basic example -- 1}

% \begin{frame}[fragile]
% \frametitle{Title}
% \framesubtitle{Subtitle}

Output Structure:\\

\begin{itemize}
  \item Version number
  \item User input (print file again)
  \item comprehensive keyword(parameter) list 
  \item Calculated output (depends on output keywords)
  \begin{itemize}
    \item Basic reaction parameters (Energy, channels, \dots)
    \item Cross Sections summary: per channel, residual nucleus, ...
    \pause
    \item Discrete Levels \& Spins included 
    \item Angular distributions
    \item Gamma-ray strength function
    \item Level desity
    \item [...]
  \end{itemize}
\end{itemize}


\end{frame}

%%%%%%%%%%%%%%%%%%%%%%%%%%

\begin{frame}{Task 2: $^{208}\mathrm{Pb}$}

Tasks: $^{208}\mathrm{Pb}$
\begin{itemize}
  \item Calculate (n,$\gamma$), (n,n$^\prime$), (n,2n) cross-sections
  \item Incident energy range: 0 - 20 MeV
  \item Visualize and compare to experimental data and evaluations
\end{itemize}

\quad
\pause

Tips:
\begin{itemize}
  \item Create a folder for each input file
  \item Use a file or grid with \pythoninline{energy}, eg. \pythoninline{energy n0-20.grid}\footnote{Computation time with this option for me: 35 sec}
  \item Extract cross-sections with \pythoninline{fileresidual y}
  \item Comapare to experiental cross-sections from \textit{EXFOR} (see next slide(s))
  \item Compare to evaluated cross-sections from \textit{JENDL} and \textit{ENDFB}
\end{itemize}

\end{frame}

\begin{frame}[fragile]{Output -- Example 2}

Cross-sections in rpZZZAAA.tot\\
For $^{208}\mathrm{Pb}$(n,$\gamma$) we look at 
\pythoninline{rp082209.tot}

\begin{scriptsize}
\begin{columns}
\begin{column}{.7\textwidth}
\begin{bashcodebg}
\## n + 208Pb: Production of 209Pb - Total
# Q-value    = 3.93737E+00 mass= 208.981090
# E-threshold= 0.00000E+00
# # energies =    78
#     E          xs
 1.00000E-11 1.15688E+04
 2.53000E-08 2.30000E+02
     ...         ...
\end{bashcodebg}
\end{column}

\begin{column}{.29\textwidth}
\vspace{\baselineskip}\linebreak
$\bf E$ in MeV, $\bf xs (\sigma)$ in mb 
\end{column}
\end{columns}
\end{scriptsize}

\quad \\

For $\langle \sigma v \rangle$ we look at 
\pythoninline{astrorate.g}

\begin{scriptsize}
\begin{bashcodebg}
\## Reaction rate for 208Pb(n,g) at <E>= 0.03000 MeV 
              # (Excited States Contribution : n)
#    T       Rate       MACS
  0.3481 3.37251E+04 2.33280E-01
\end{bashcodebg}
\end{scriptsize}
\textbf{MACS} again in mb

\end{frame}

%%%%%%%%%%%%%%%%%%%%%

% \begin{frame}{Discussion}

% Discussion: MACS for $^{208}\mathrm{Pb}$(n,$\gamma$) with $kT$=30 keV
% In one of the previous slides you were asked to calculate the MACS ($\langle \sigma v \rangle$) with \pythoninline{astroE 0.03}, where $kt=0.03\,\mathrm{MeV}$ and compare to \textit{KADONIS}.\\


% The comparison of the results from Talys to KADONIS is difficult in this case. Why?\\


% Tip: You may plot the results from Talys on a loglog scale together with experimental data and a Maxwellian with $kT$=30 keV.

% % \begin{figure}[tbh]
% % \centering
% % \includegraphics[width=\linewidth]{code/ex2}
% % \caption{$^{208}\mathrm{Pb}$(n,$\gamma$) compared to experiment}
% % %\label{fig:EnergyRegionLevelD}
% % \end{figure}

% \end{frame}

%%%%%%%%%%%%%%%%%%%%%

%%%%%%%%%%%%%%%%%%%%%

\begin{frame}{(n,$\gamma$) cross-sections vs experiment}

\begin{figure}[tbh]
\centering
\includegraphics[width=\linewidth]{code/ex2}
\caption{$^{208}\mathrm{Pb}$(n,$\gamma$) compared to experiment}
%\label{fig:EnergyRegionLevelD}
\end{figure}

\end{frame}

%%%%%%%%%%%%%%%%%%%%%

\begin{frame}{How to retrieve data from EXFOR (experiments) 1 ?}

\url{https://www-nds.iaea.org/exfor/}

\begin{figure}[b]
	\centering
	\makebox[\linewidth][c]{%
	\begin{subfigure}{.7\linewidth}
		\hspace{1.5cm}
		\includegraphics[width=1\textwidth]{figs/exfor}
	\end{subfigure}%
	\quad
	\begin{subfigure}{.7\linewidth}
		\includegraphics[width=1\textwidth]{figs/exfor-select}
	\end{subfigure}%
	}
\end{figure}
	
\end{frame}

%%%%%%%%%%%%%%%%%%%%%

\begin{frame}{How to retrieve data from EXFOR (experiments) 2 ?}
	
	\begin{figure}[b]
		\centering
		\makebox[\linewidth][c]{%
			\hspace{1.5cm}
			\begin{subfigure}{.7\linewidth}
				\includegraphics[width=1\textwidth]{figs/exfor-data1}
			\end{subfigure}%
			\quad
			\begin{subfigure}{.7\linewidth}
				\includegraphics[width=1\textwidth]{figs/exfor-data2}
			\end{subfigure}%
		}
	\end{figure}
	
\end{frame}

%%%%%%%%%%%%%%%%%%%%%

\begin{frame}{How to retrieve data from ENDF(B) (evaluations)?}
	
\url{https://www-nds.iaea.org/exfor/endf.htm}
\\or use JANIS:
\url{http://www.oecd-nea.org/janis/}

	\begin{figure}[b]
		\centering
		\makebox[\linewidth][c]{%
			\begin{subfigure}{.5\linewidth}
				\includegraphics[width=1\textwidth]{figs/endf}
			\end{subfigure}%
			\quad
			\begin{subfigure}{.5\linewidth}
				\includegraphics[width=1\textwidth]{figs/endfb-data}
			\end{subfigure}%
		}
		\makebox[\linewidth][c]{%
			\begin{subfigure}{.5\linewidth}
				\includegraphics[width=1\textwidth]{figs/endf-select}
			\end{subfigure}%
			\quad
			\makebox[.5\linewidth][c]{}
		}
	\end{figure}
	
\end{frame}


%%%%%%%%%%%%%%%%%%%%%

\begin{frame}[fragile]{Effect of different level density models}

Task:
\begin{itemize}
  \item Investigate effect of CT+FG (ld 1; default), BSFG (ld 2), HF (ld 5)
  \item Plot the total LD for the different models
  \item Compare (plot) $\sigma_{(\mathrm{n},\gamma)}$
\end{itemize}
Output level density with \pythoninline{outdensity y}\\

% WORKAROUND
\begingroup % WORKAROUND
\catcode`\!=\active   % WORKAROUND
\def!#1!{\colorbox{yellow}{#1}}  % WORKAROUND
\def¤#1¤{\colorbox{black!30}{[...]}}  % WORKAROUND
\begin{tiny}
\begin{bashcodebg}
\##
# General
#
projectile n
element pb
mass 208
energy energies
#
# Parameters
# 
# default ldmodel = 1
!ldmodel 1!
#
# Output
#
fileresidual y
outdensity y
\end{bashcodebg}
\end{tiny}
\endgroup

\end{frame}

%%%%%%%%%%%%%%

\begin{frame}[fragile]{Effect of different level density models}

Investigation of the output file $\rightarrow$ Find LD
% WORKAROUND
\begingroup % WORKAROUND
\catcode`\!=\active   % WORKAROUND
\def!#1!{\colorbox{yellow}{#1}}  % WORKAROUND
\def¤#1¤{\colorbox{black!30}{[...]}}  % WORKAROUND
\begin{tiny}
\begin{bashcodebg}[escapeinside=||]
|¤¤|
Level density parameters for Z= 82 N=127 |!(209Pb)!|

Model: Gilbert-Cameron          
Collective enhancement: no

a(Sn)           :   9.02372
Experimental D0 :         400000.00 eV +-     80000.00000
Theoretical D0  :         612808.44 eV
|¤¤|
Matching Ex     :   6.28452
Temperature     :   0.75450
E0              :   0.92010
Adj. pair shift :   0.00000
Discrete sigma  :   3.53026
Sigma (Sn)      :   4.65249

Level density |!per parity!| for ground state
(Total level density also per parity)

   Ex     a    sigma   |!total!|   JP=  0.5  JP=  1.5  JP=  2.5  JP=  3.5  JP=  4.5  JP=  5.5  JP=  6.5  JP=  7.5  JP=  8.5

  0.25  7.661  3.530 |!2.726E-01!| 2.102E-02 3.727E-02 4.574E-02 4.605E-02 4.012E-02 3.097E-02 2.144E-02 1.343E-02 7.637E-03
|¤¤|
\end{bashcodebg}
\end{tiny}
\endgroup

\end{frame}

%%%%%%%%%%%%%%%%%


\begin{frame}{Effect of different level density models}

\begin{figure}[tbh]
\centering
\includegraphics[width=0.9\linewidth]{code/ex2_LD}
\caption{$^{208}\mathrm{Pb}$(n,$\gamma$) for different LD models}
%\label{fig:EnergyRegionLevelD}
\end{figure}
\end{frame}

%%%%%%%%%%%%%%%

\begin{frame}{Effect of $\gamma$SF}

\begin{figure}[tbh]
\centering
\includegraphics[width=0.9\linewidth]{code/ex2_xsgSF}
\caption{$^{208}\mathrm{Pb}$(n,$\gamma$) shifting the energy centroid of the GDR}
%\label{fig:EnergyRegionLevelD}
\end{figure}

\end{frame}


%%%%%%%%%%%%%%

% \begin{frame}[fragile]{Effect of $\gamma$SF}

% Investigation of the output file: Default; output with \pythoninline{outgamma y}

% \begin{minipage}{1.2\textwidth}
% % WORKAROUND
% \begingroup % WORKAROUND
% \catcode`\!=\active   % WORKAROUND
% \def!#1!{\colorbox{yellow}{#1}}  % WORKAROUND
% \def¤#1¤{\colorbox{black!30}{[...]}}  % WORKAROUND
% \begin{tiny}
% \begin{bashcodebg}[escapeinside=||]
% |¤¤|
% #### GAMMA STRENGTH FUNCTIONS, TRANSMISSION COEFFICIENTS AND CROSS SECTIONS ######

%  Gamma-ray information for Z= 82 N=127 (209Pb) 

%  S-wave strength function parameters:

%  Exp. total radiative width=   0.06000 eV +/- 0.00000 Theor. total radiative width=        0.07018 eV
%  Exp. D0                   = 400000.00 eV +/-80000.00 Theor. D0                   =   612808.43750 eV
%  Theor. S-wave strength f. =   0.00115E-4
%  Normalization factor      =   0.85495

%  Gamma-ray strength function model for E1: Kopecky-Uhl              

%  Gamma-ray strength function model for M1: RIPL-2                   

%  Normalized gamma-ray strength functions and transmission coefficients for l= 1

%  Giant resonance parameters :

%  sigma0(M1) =   0.909       sigma0(E1) = 591.220    PR: sigma0(M1) =   0.000       sigma0(E1) =   0.000
%       E(M1) =   6.909            E(E1) =  13.714    PR:      E(M1) =   0.000            E(E1) =   0.000
%   gamma(M1) =   4.000        gamma(E1) =   3.863    PR:  gamma(M1) =   0.000        gamma(E1) =   0.000
%       k(M1) =   8.67373E-08      k(E1) =   8.67373E-08

%      E       f(M1)        f(E1)        T(M1)        T(E1)

%    0.001  0.00000E+00  1.28340E-08  0.00000E+00  5.99831E-16
%    0.002  9.46841E-13  1.28312E-08  4.75934E-20  4.79894E-15
% |¤¤|
% \end{bashcodebg}
% \end{tiny}
% \endgroup
% \end{minipage}

% \end{frame}

%%%%%%%%%%%%%%%%%



\end{document}